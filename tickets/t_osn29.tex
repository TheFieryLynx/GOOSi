\subsection{OSN 30 Квадратурные формулы прямоугольников, трапеций и парабол.}

\textbf{Задача:} Вычислить определенный интеграл $I = \int\limits_a^b f(x)dx$.

Не всегда можно посчитать аналитически, но есть универсальные алгоритмы -- формулы численного интегрирования (квадратурные формулы). 
В них интеграл заменяется конечной суммой:

$ \int\limits_a^b f(x)dx \approx\displaystyle\sum_{k=0}^N C_k f(x_k)$ --- \textbf{квадратурная формула}, 
$C_k$ --- \textbf{коэффициенты квадратурной формулы}, 
$x_k \in [a, b]$ --- \textbf{узлы квадратурной формулы}.

$\psi = \int\limits_a^b f(x)dx - \sum_{k=0}^N C_k f(x_k) $ --- \textbf{погрешность квадратурной формулы}. 

Введем на $[a,b]$ равномерную сетку $\omega_n = \{x_i =a+ih,~i\in[0,N],~N_h=b-a \}$.

$\int\limits_a^b f(x)dx = \displaystyle\sum_{i=0}^N\int\limits_{x_{i-1}}^{x_i} f(x)dx$. Строим квадратурные формулы для $\int\limits_{x_{i-1}}^{x_i} f(x)dx$.

\textbf{Формула прямоугольников.}

$$ \int\limits_{x_{i-1}}^{x_i} f(x)dx \sim f \left( x_{i-\frac{1}{2}} \right) h $$

$\psi_i = \int\limits_{x_{i-1}}^{x_i} f(x)dx - f \left( x_{i-\frac{1}{2}} \right)h = 
\int\limits_{x_{i-1}}^{x_i} \left( f(x) - f\left(x_{i-\frac{1}{2}} \right) \right)dx = 
\int\limits_{x_{i-1}}^{x_i} 
\Biggl( 
    f(x_{i-\frac{1}{2}}) + 
    (x - x_{i-\frac{1}{2}})f'(x_{i-\frac{1}{2}}) + 
    \frac{\left(x - x_{i-\frac{1}{2}}\right)^2}{2}f''(\xi) 
    \Biggr|_{\xi \in [x_{i-1};x_i]} - 
    f(x_{i-\frac{1}{2}}) 
\Biggr) dx$

$|\psi_i| \leqslant 
M_{2i} \int\limits_{x_{i-1}}^{x_i} \frac{ \left(x - x_{i-\frac{1}{2}} \right)^2}{2}dx = 
\frac{h^3}{24} M_{2i}$, где $ M_{2i} = \displaystyle\max_{x\in[x_{i-1};x_i]}|f''(x)|$

$\int\limits_a^b f(x)dx \sim \displaystyle\sum_{i=0}^N f(x_{i-\frac{1}{2}})h$

$\psi = \displaystyle\sum_{i=0}^N \psi_i \leqslant \displaystyle\sum_{i=0}^n \frac{h^3}{24} M_{2i}\leqslant \frac{M_2Nh^3}{24} = \frac{M_2h^2(b-a)}{24} \implies \psi = O(h^2)$

\textbf{Формула трапеций.}

$$ \int\limits_{x_{i-1}}^{x_i} f(x)dx \sim \frac{f(x_{i-1}) +f (x_i)}{2}h $$
получается путем замены $f(x)$ интерполяционным многочленом первой степени, построенным по узлам $x_{i-1},~x_i$.

$$L_{1i} = \frac{\left((x-x_{i-1})f(x_i)-(x-x_i)f(x_{i-1})\right)}{h}$$

$$f(x) - L_{1i} = \frac{( x-x_{i-1})(x-x_i )}{2}f''(\xi_i(x))$$

$\left|\psi_j\right| =  \int\limits_{x_{i-1}}^{x_i} f(x)dx - \frac{f(x_{i-1}) +f (x_i)}{2}h = \int\limits_{x_{i-1}}^{x_i} (f(x) - L_{1i})dx =$ \\
$= \int\limits_{x_{i-1}}^{x_i} \frac{(x-x_{i-1})(x-x_i)}{2}f''(\xi_i(x)) \implies |\psi_j| \leqslant \frac{M_{3j}h^3}{12}$

$ \int\limits_a^b f(x)dx \sim \displaystyle\sum_{i=1}^N\frac{f(x_{i-1}) +f (x_i)}{2}h = $

$ = h \left(0.5f(x_0) + f(x_1) + f (x_2) + \dots + f(x_{N-1}) + 0.5f(x_N ) \right)$
--- составная формула трапеций. 

$$|\psi| \leqslant \frac{M_3h^2(b-a)}{12} = O(h^2)$$

\textbf{Формула Симпсона (парабол).}

$L_n = \displaystyle\sum_{k=0}^n L_{n_k} = \displaystyle\sum_{k=0}^n \frac{\omega(x)}{(x-x_k)\omega'(x_k)}f(x_k)$
--- интерполяционный полином в Форме Лагранжа, где

$$\omega(x) = \displaystyle\prod_{j=0}^n(x-x_j),~\omega'(x_k)=\displaystyle\prod_{\substack{j=0 \\ j\neq k}}^n (x_k - x_j),~f(x) - L_n = \frac{f^{(n+1)}(\xi(x))}{(n+1)!}\omega(x)$$

В формуле Симпсона:

$f(x) \sim L_2 \sim \frac{(x-x_{i-\frac{1}{2}})(x-x_i)}{(x_{i-1}-x_{i-\frac{1}{2}})(x_{i-1}-x_i)}f(x_{i-1}) + \frac{(x-x_{i-1})(x-x_i)}{(x_{i-\frac{1}{2}}-x_{i-1})(x_{i-\frac{1}{2}}-x_{i-1})}f(x_{i-\frac{1}{2}}) + \frac{(x-x_{i-1})(x-x_{i-\frac{1}{2}})}{(x_i-x_{i-1})(x_i-x_{i-\frac{1}{2}})}f(x_{i}) = \frac{2}{h^2}((x-x_{i-\frac{1}{2}})(x-x_i)f(x_{i-1})-2(x-x_{i-1})(x-x_i)f(x_{i-\frac{1}{2}}) + (x-x_{i-1})(x-x_{i-\frac{1}{2}})f(x_i)),~ \forall x\in[x_{i-1},x_i]$

$$\int\limits_{x_{i-1}}^{x_i} L_{2_i}(x) dx = \frac{h}{6}(f(x_{i-1}) + 4f(x_{i-\frac{1}{2}}) + f(x_i)) $$
$$\implies \int\limits_{a}^{b} f(x) dx \approx \displaystyle\sum_{i=1}^{N}\frac{h}{6}(f(x_{i-1}) + 4f(x_{i-\frac{1}{2}}) + f(x_i))$$
$$\int\limits_{a}^{b} L_{2}(x) dx \approx \frac{h}{6}(f_0 + f_N + 2(f_1 + \dots + f_{N-1}) + 4(f_{\frac{1}{2}} + \dots + f_{N - \frac{1}{2}})) $$

$$ \psi_i \leqslant \frac{M_{4i}h^4}{2880},~|\psi| \leqslant \frac{M_4h^4(b-a)}{2880} \implies \psi = O(h^4)$$

\todo{сюда бы картинки разбиенитя площади под кривой для разных методов}

% -------- source --------
\bigbreak
[\cite[page 69-96]{replace_me}]