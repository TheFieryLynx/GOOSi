\subsection{DOP 13 Качество программного обеспечения и методы его контроля. Тестирование и другие методы верификации.}

\textbf{Основные определения}
\begin{itemize}
    \item \textbf{Качество ПО} в стандарте ISO 9126 --- вся совокупность его характеристик, относящихся к возможности удовлетворять высказанные или подразумеваемые потребности всех заинтересованных лиц.
    \item Внутреннее качество связано с характеристиками ПО самого по себе, без учета его поведения; внешнее качество характеризует ПО с точки зрения его поведения; качество ПО при использовании в различных контекстах --- качество, которое ощущается пользователями при конкретных сценариях работы ПО.
    \item \textbf{Верификация} означает проверку того, что ПО разработано в соответствии со всеми требованиями к нему, или что результаты очередного этапа разработки соответствуют ограничениям, сформулированным на предшествующих этапах.
    \item Валидация --- проверка того, что сам продукт правилен, то есть подтверждение того, что он действительно удовлетворяет потребностям и ожиданиям пользователей, заказчиков и других заинтересованных сторон.
    \item \textbf{Методы обеспечения качества} представляют собой техники, гарантирующие достижение определенных показателей качества при их применении.
\end{itemize}

\textbf{Методы верификации}

\begin{itemize}
    \item Методы и техники выяснения свойств ПО во время его работы.
    Это, прежде всего, все виды тестирования.
    \item Методы и Техники определения качества на основе симуляции работы ПО с помощью моделей: проверка на моделях (model checking), прототипирование (макетирование для оценки качества принимаемых решений).
    \item Методы и Техники выявления нарушений формализованных правил построения исходного кода ПО, проектных моделей и документации: инспектирование кода.
    \item Методы и Техники обычного или формализованного анализа проектной документации и исходного кода для выявления их свойств: методы анализа архитектуры ПО.
\end{itemize}

\textbf{Виды тестирования}

\begin{itemize}
    \item Стрессовое (нагрузочное) тестирование --- проверяет производительность ПО в условиях повышенных нагрузок.
    \item Тестирование черного ящика (тестирование на соответствие) --- проверка требований спецификаций, стандартов, ограничений.
    \item Функциональное тестирование --- его частный случай, проверка требований к функциональности.
    \item Аттестационное тестирование --- для получения документа о соответствии.
    \item Тестирование белого ящика --- на основе знаний о структуре системы.
    \item Тестирование на отказ --- попытка вывести систему из строя в том числе некорректными данными.
    \item Тестирование с помощью набора мутантов --- программ, отличных от тестируемой в отдельных точках.
    \item Модульное тестирование --- проверка отдельных модулей. 
    Важная часть отладочного тестирования.
    \textbf{Предусловия} описывают для каждой операции, на каких входных данных она предназначена работать, \textbf{постусловия} --- как должны соотноситься входные данные с возвращаемыми ею результатами, \textbf{инварианты} --- определяют критерии целостности внутренних данных модуля.
    \item Интеграционное тестирование --- проверка правильности взаимодействия некоторого набора модулей друг с другом.
    \item Системное тестирование --- проверка правильности работы системы в целом, ее способности правильно решать поставленные пользователями задачи в различных ситуациях.
    \item Тестирование пользовательского интерфейса --- имитация действий пользователей над элементами этого интерфейса. Частные случаи --- тестирование графического интерфейса, тестирование интерфейса Web-приложений.
\end{itemize}
% -------- source --------
\bigbreak
[\cite[page 69-96]{replace_me}]