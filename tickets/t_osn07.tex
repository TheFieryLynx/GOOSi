\textbf{\LARGE osn 7. Производная функции комплексногопеременного. Условия Коши-Римана. Аналитическая  функция.}

\textbf{Дифференцируемость функции.}

$\mathLet f(z)$ определена на $U_{\delta}(z_0),~\delta > 0$. Если $\exists \displaystyle\lim_{z\to z_0} \frac{f(z) - f(z_0)}{z - z_0}$,
то этот предел называют \textbf{производной функции} $f(z)$ в точке $z_0$ и обозначают $f'(z_0)$.

Положим $$f(z) = u(x, y) + iv(x, y),~z = x + iy,~z_0 = x_0 + iy_0,$$
$$\Delta f = f(z)-f(z_0),~\Delta f = \Delta u + i\Delta v,~\Delta z = z- z_0,~\Delta z = \Delta x + i\Delta y,$$
$$\Delta u = u(x, y)-u(x_0, y_0),~\Delta v = v(x, y)-v(x_0, y_0),~\Delta x = x-x_0,~\Delta y = y- y_0.$$

Функция $f(z)$ называется \textbf{дифференцируемой} в точке $z_0$, если приращение $\Delta f = A\Delta z + \overline{o}(1)\Delta z$, где $\displaystyle\lim_{\Delta z\to0} \overline{o}(1) = 0$, а $A$ не зависит от $\Delta z$.

\textbf{Условия Коши-Римана:}
$u'_x = v'_y,~u'_y = -v'_x.$

\textbf{Критерий дифференцируемости функции комплексного переменного.}
Для того, чтобы функция $f(z) = u(x,y)+iv(x,y)$ была дифференцируема в точке $z_0 =x_0+iy_0$, необходимо и достаточно, чтобы функции $u(x,y)$ и $v(x,y)$ были дифференцируемы в точке $(x_0, y_0)$ и выполнялись условия Коши-Римана.

\begin{proof}
($\impliedby$) Предположим, что функции $u(x,y)$ и $v(x,y)$ дифференцируемы в точке $(x_0, y_0)$ и выполнены условия Коши-Римана.

Пусть $u'_x =a=v'_y,~u'_y =-b=-v'_x$. В силу дифференцируемости функций $u$ и $v$ имеем:
$$\Delta u = a\Delta x - b\Delta y + \overline{o}_1(1)\Delta x + \overline{o}_2(1)\Delta y,$$
$$\Delta v = b\Delta x + a\Delta y + \overline{o}_3(1)\Delta x + \overline{o}_4(1)\Delta y,$$
$$\lim\limits_{\substack{\Delta x\to0 \\ \Delta y\to0}} \overline{o}_i(1)=0,~i=1,2,3,4.$$

Отсюда $\Delta f = \Delta u+\Delta iv = a\Delta x-b\Delta y + \overline{o}_1(1)\Delta x + \overline{o}_2(1)\Delta y + ib\Delta x + ia\Delta y + i\overline{o}_3(1)\Delta x + i\overline{o}_4(1)\Delta y = $

$ = \Delta x(a + ib) + \Delta y(-b + ia) + \Delta x(\overline{o}_1(1) + i\overline{o}_3(1)) + \Delta y(\overline{o}_2(1) + i\overline{o}_4(1)) = $

$ = (a + ib)(\Delta x + \Delta iy) + \Delta x(\overline{o}_1(1) + i\overline{o}_3(1)) + \Delta y(\overline{o}_2(1) + i\overline{o}_4(1))$.

Тем самым
$$\frac{\Delta f}{\Delta z} = a + ib + \frac{\Delta x}{\Delta z}(\overline{o}_1(1) + i\overline{o}_3(1)) + \frac{\Delta y}{\Delta z}(\overline{o}_2(1) + i\overline{o}_4(1)) .$$
Так как $|\Delta x| \leqslant |\Delta z|,~|\Delta y| \leqslant |\Delta z|,$ то
$$\lim\limits_{\Delta z\to0} \frac{\Delta f}{\Delta z} =a+ib=f'(z_0)=u'_x +iv'_x =v'_y-iu'_y =u'_x-iu'_y =v'_y +iv'_x $$.

($\implies$) Пусть $f(z)$ дифференцируема в точке $z_0$, тогда

$\Delta f = \Delta u+\Delta iv = f'(z_0) \Delta z + \overline{o}(1)\Delta z=f'(z_0)(\Delta x + i\Delta y)+\overline{o}(1)(\Delta x + i\Delta y),~\displaystyle\lim_{\Delta z\to0} \overline{o}(1) = 0$

Пусть $f'(z_0) = a+ib,~\overline{o}(1) = \alpha_1 + i\alpha_2$, тогда:

$\Delta u+\Delta iv = (a+ib)(\Delta x + i\Delta y)+(\alpha_1 + i\alpha_2)(\Delta x + i\Delta y) = (a\Delta x - b\Delta y)+i(a\Delta y + b\Delta x) +(\alpha_1\Delta x - \alpha_2\Delta y) + i(\alpha_1\Delta y + \alpha_2\Delta x)$

Отдельно для действительной и мнимой частей:
$$\Delta u = a\Delta x - b\Delta y + \alpha_1\Delta x - \alpha_2\Delta y$$
$$\Delta v = b\Delta x + a\Delta y + \alpha_1\Delta y + \alpha_2\Delta x$$
Отсюда, в силу того, что $\lim\limits_{\substack{\Delta x\to0 \\ \Delta y\to 0}} \alpha_1 = 0,~\lim\limits_{\substack{\Delta x\to0 \\ \Delta y\to 0}} \alpha_2 = 0$, следует, что
функции $u, v$ дифференцируемы в точке $(x_0,y_0)$ и удовлетворяют условиям Коши-Римана.
\end{proof}

\textbf{Аналитическая функция.}

\textbf{Достаточное условие существования} $f'(z)$.
Если $f(z) = u(x, y) + iv(x, y)$ и функции $u(x, y),~v(x, y)$ в области $D$ имеют непрерывные частные производные первого порядка и выполняются условия Коши-Римана, то функция $f(z)$ дифференцируема в области $D$ и $f'(z) \in C(D)$.

Функция $f(z)$ называется \textbf{аналитической} в области $D$, если она в каждой точке области $D$ имеет производную $f'(z) \in C(D)$.

\textit{Свойства аналитических функций:}
\begin{enumerate}
    \item Аналитическая в области $D$ функция непрерывна в $D$.
    \item Сумма и произведение аналитических функций являются аналитическими функциями. Частное $\dfrac{f(z)}{g(z)}$ двух аналитических функций является аналитической всюду, где $g(z) \neq 0$.
    \item Если $f(z), g(w)$ - аналитические функции, то сложная функция $t = g(f(z))$ является аналитической функцией переменного $z$.
    \item Если функция $w=f(z)$ аналитична в области $D$, и в окрестности точки $z_0 \in D, f'(z) \neq 0$, то в окрестности точки $w_0 = f(z_0) \in W$ определена обратная функция $z=g(w)$, являющаяся аналитической функцией переменного $w$. При этом $f'(z_0) = \dfrac{1}{g'(z_0)}$.
\end{enumerate}

Пусть $f(z)$ - комплекснозначная функция комплексного переменного $z$, $u(x,y) = Re f(z), v(x, y) = Im f(z), z = x + iy$. Интегралом по кривой $L$, лежащей в области опредения $f$, называется $\int_L f(z) dz = \int_L (u(x,y) dx - v(x, y) dy) + i \int_L (u(x,y) dy + v(x, y) dx)$.

\textbf{Теорема.} 
Пусть заданная в односвязной области $D$ функция $f(z)$ -однозначная аналитическая функция. Тогда интеграл от $f(z)$ по любому замкнутому контуру $L$, лежащему в $D$, равен нулю: $\oint_I f(z) dz = 0$.

\textbf{Теорема.}
Пусть заданная в односвязной области $D$ функция $f(z)$ непрерывна, а интеграл от $f(z)$ по любому замкнутому контуру $L$, лежащему в $D$, равен нулю. Тогда $f(z)$ является аналитической функцией в области $D$.

