dop 1. Теорема Поста о полноте систем функций в алгебре логики.

dop 2. Графы,  деревья,  планарные графы;  их свойства.  Оценка числа деревьев.

dop 3. Логика 1-го порядка.  Выполнимость и общезначимость. Общая схема метода резолюций.

dop 4. Логическое  программирование. Декларативная семантика и операционная семантика;  соотношение между ними.  Стандартная стратегия выполнения логических программ.

dop 5. Сортировка.  Простейшие  алгоритмы – сортировка  выбором,  вставками,  обменом.  Оценка  сложности алгоритмов сортировки. Быстрая сортировка и ее сложность в среднем и в наихудшем случаях.

dop 6. Язык  ассемблера  как  машиннозависимый  язык  низкого  уровня.  Организация  ассемблерной  программы, секции кода и данных (на примере ассемблера nasm или masm). Основные  этапы подготовки к счёту ассемблерной программы: трансляция, редактирование внешних связей (компоновка), загрузка.

dop 7. Операционные системы. Управление оперативной памятью в вычислительной системе. Алгоритмы и методы организации и управления страничной оперативной памятью.

dop 8. Зависимости  в  реляционных  отношениях:  функциональные,  многозначные,  проекции/соединения. Проектирование реляционных БД на основе принципов нормализации отношений. Нормальные формы.

dop 9. Закон Амдала, его следствия. Граф алгоритма. Критический путь графа алгоритма, ярусно-параллельная форма графа алгоритма. Этапы решения задач на параллельных вычислительных системах.

dop 10. Глобальные и локальные модели освещения в компьютерной графике. Модель Фонга.

dop 11. Классификация  языков,  определяемых  конечными  автоматами,  регулярными  выражениями  и праволинейными грамматиками. Эквивалентность и минимизация конечных автоматов.

dop 12. Функции FIRST и FOLLOW. LL(l)-грамматики. Конструирование таблицы предсказывающего анализатора.

dop 13. Жизненный  цикл  программного  обеспечения  (ПО).  Основные  виды  деятельности  при  разработке  ПО. Каскадная и итерационная модели жизненного цикла.

dop 14. Качество программного обеспечения и методы его контроля. Тестирование и другие методы верификации.

dop 15. Основные  понятия  криптографии.  Односторонняя  функция  с  секретом.  Протокол  Диффи-Хеллмана выработкиобщего секретного ключа по открытому каналу связи.

dop 16. Основные  принципы  построения  и  архитектура  сети  Интернет.  Алгоритмы  и  протоколы  внешней  и внутренней маршрутизации. Явление перегрузки и методы борьбы с ней.

dop 17. Теоретические основы передачи данных, физический уровень стека протоколов. Системы передачи данных Ethernet и Wi-Fi: алгоритмы работы, управление множественным доступом к каналу

dop 18. Базисные типы данных в языках программирования. Основные проблемы, связанные с базисными типами и способы их решения в различных языках. Понятие абстрактного типа данных и способы его реализации в современных языках программирования.

dop 19. Понятие  о  парадигме  программирования.  Основные  парадигмы  программирования.  Языки  и  парадигмы программирования.

dop 20. Основные  характеристики  функциональных  языков  программирования.  Использование  понятий функционального  программирования  (замыкания,  анонимные функции)  в  современных  объектно-ориентированных языках.

dop 21. Синхронизация  в  распределенных  системах.  Синхронизация  времени.  Логические  часы.  Выборы координатора. Взаимное исключение. Координация процессов.

dop 22. Отказоустойчивость  в  распределенных  системах.  Типы  отказов.  Фиксация  контрольных  точек  и восстановление после отказа. Репликация и протоколы голосования. Надежная групповая рассылка.

dop 23. Распределенные файловые системы. Доступ к директориям и файлам. Семантика одновременного доступа к одному файлу нескольких процессов. Кэширование и размножение файлов.

dop 24. Промежуточные  представления  программы:  абстрактное  синтаксическое  дерево;  последовательность трехадресных инструкций. Базовые блоки и граф потока управления.

dop 25. Локальная  оптимизация при  компиляции  программы. Ориентированный  ациклический  граф  и  метод нумерации значений.

dop 26. Глобальная оптимизация при компиляции программы. Построение передаточных функций базовых блоков. Монотонные и дистрибутивные передаточные функции. Метод неподвижной точки и его применение для нахождения достигающих определений.

dop 27. Постановка задачи дискретной оптимизации. Метод ветвей и границ. Задача целочисленного линейного программирования.

dop 28. Комбинаторные методы нахождения оптимального пути в графе.

dop 29. Потоки в сетях. Алгоритм построения максимального потока. Оценка сложности алгоритма.